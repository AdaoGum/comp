\documentclass{compiladores}

\renewcommand{\flecha}{\rightarrow}
\newcommand{\flech}{$\rightarrow$}

\begin{document}

\begin{center}
{\LARGE Material de Apoio \#1} \\
Transformações sobre Gramáticas Livres de Contexto
\end{center}

Transformações sobre gramáticas livres de contexto (GLC) podem ser
necessárias por várias razões como, por exemplo, possibilitar o
funcionamento de uma determinada classe de algoritmos de
reconhecimento sintático para uma gramática. Este material de apoio
detalha o funcionamento de três transformações: eliminação de
produções vazias; eliminação da recursividade à esquerda e, por fim, a
mais simples dentre elas, fatoração.

\section{Eliminação de produções vazias}

Produções vazias em uma gramática livre de contexto são aquelas
produções da forma $A \flecha \epsilon$, ou ainda situações onde $B
\flecha AAA$ e onde o não-terminal A pode derivar para vazio. Nestas
duas situações, tanto A quanto B podem derivar para vazio: A de forma
direta; e B porque os três não-terminais A podem, conjuntamente,
derivar para vazio. Vemos a seguir dois métodos para a eliminação de
produções vazias. O primeiro, baseado na intuição e no conhecimento da
gramática, serve mais como uma motivação para o segundo método, mais
genérico e automático.

\subsection{Método manual: seguindo a intuição}


A remoção das produções vazias pode eventualmente ser feita utilizando
a intuição e o conhecimento sobre a gramática. Veja, por exemplo, a
gramática seguinte que reconhece chamada de funções em uma linguagem
imperativa:

\begin{center}
\begin{tabular}{lll}
Chamada\_de\_função   & \flech & {\bf id (} Argumentos\_opcionais {\bf )}\\
Argumentos\_opcionais & \flech & Lista\_argumentos \\
Argumentos\_opcionais & \flech & $\epsilon$ \\
Lista\_argumentos     & \flech & Arg \\
Lista\_argumentos     & \flech & Arg {\bf ,} Lista\_argumentos \\
\end{tabular}
\end{center}

No exemplo acima, a produção vazia Argumentos\_opcionais \flech
$\epsilon$ poderia ser facilmente removida alterando ligeiramente a
gramática. Isto pode ser feito neste caso porque os argumentos da
função são opcionais. Podemos portanto adicionar uma regra de produção
para o não-terminal Chamada\_de\_função sem o não-terminal entre os
dois parênteses, removendo a produção vazia da gramática. O resultado
final fica assim:

\begin{center}
\begin{tabular}{lll}
Chamada\_de\_função   & \flech & {\bf id (} Argumentos\_opcionais {\bf )}\\
Chamada\_de\_função   & \flech & {\bf id ( )}\\
Argumentos\_opcionais & \flech & Lista\_argumentos \\
Lista\_argumentos     & \flech & Arg \\
Lista\_argumentos     & \flech & Arg {\bf ,} Lista\_argumentos \\
\end{tabular}
\end{center}

\subsection{Método automático}

Embora possamos seguir a nossa intuição para remover produções vazias
de uma gramática, podemos observar várias situações mais complexas
onde a aplicação do nosso conhecimento da gramática é insuficiente
para remover as produções vazias. Nestas situações, convém aplicar um
algoritmo automático para removê-las. Para ilustrar o funcionamento do
algoritmo, vejamos o exemplo abaixo baseado na gramática seguinte:

\begin{center}
\begin{tabular}{lll}
A & \flech & B {\bf z} B \\
B & \flech & {\bf b} \\
B & \flech & $\epsilon$ \\
\end{tabular}
\end{center}

Ao remover a produção $B \flecha \epsilon$, somos obrigados a replicar
as produções que contém $B$ no lado direito. No exemplo, a produção
que contém $B$ do lado direito é a produção $A \flecha B {\bf z}
B$. Uma vez que B pode se tornar vazio, a única forma de remover com
segurança a produção vazia é considerar todas as combinações possíveis
para a forma sentencial $B{\bf z}B$, ou seja, adicionar as regras $A
\flecha {\bf z} \vert B {\bf z} \vert {\bf z} B \vert B {\bf z} B$ que
representam todas as formas sentenciais válidas a partir de A sabendo
que a produção vazia a partir de B foi removida. Sendo assim a
gramática resultante -- capaz de gerar exatamente a mesma linguagem da
gramática original com a produção vazia -- termina sendo da seguinte forma:

\begin{center}
\begin{tabular}{lll}
A & \flech & {\bf z} \\
A & \flech & {\bf z} B \\
A & \flech & B {\bf z} \\
A & \flech & B {\bf z} B \\
B & \flech & {\bf b} \\
\end{tabular}
\end{center}

Podemos generalizar o exemplo acima.  Se B aparece $n$ vezes em alguma
produção $P$ qualquer, esta produção $P$ será replicada $2^n$ vezes,
cada qual com uma combinação diferente. Esta descrição genérica pode
ser posta em funcionamento através do seguinte algoritmo, dividido em
três passos, que considera que a gramática inicial é $G = (N, T, P,
S)$, onde $N$ são os símbolos não-terminais, $T$ os terminais, $P$ as
produções e $S$ o símbolo não-terminal inicial:

\begin{enumerate}
\item {\bf Reunir todos os não-terminais que geram $\epsilon$}

  Para obter todos os não-terminais que geram $\epsilon$ e reuní-los
  em um conjunto identificado por $N_\epsilon$, devemos inicializar
  $N_\epsilon$ com todos os não-terminais que derivam
  \emph{diretamente} para vazio. Esta configuração inicial é
  representada por $N_\epsilon = \{ A \vert A \flecha \epsilon \}$.

  Para identificar todos os não-terminais que derivam
  \emph{indiretamente} para vazio, devemos repetir a regra $N_\epsilon
  = N_\epsilon \cup \{ X\ |\ X \rightarrow X_1...X_n \in P$ tal que
  $X_1...X_n \in N_\epsilon \}$. Isso significa que devemos olhar para
  todas as produções da gramática tentando identicar aquelas cujo
  corpo deriva inteiramente para vazio. Caso isto ocorra, adicionamos
  em $N_\epsilon$ o não-terminal da cabeça da produção sob
  análise. Este processo é repetido até que $N_{\epsilon}$ não aumente
  mais de tamanho.

\item {\bf Construir um conjunto de produções sem produções vazias}

  Neste passo, criamos uma outra gramática sem produções vazias. Esta
  gramática é $G_1 = (N, T, P_1, S)$, onde $N$ são os símbolos
  não-terminais, $T$ os terminais, $P_1$ as produções sem derivações
  para vazio e $S$ o símbolo não-terminal inicial. $N$, $T$ e $S$ são
  idênticos aos da gramática original. Para obter $P_1$, devemos
  inicializá-lo com todas as produções de $P$ (da gramática original)
  sem as produções que derivam \emph{diretamente} para
  vazio. Portanto, $P_1$ é configurado inicialmente como $P_1 = \{ A
  \flecha \alpha \vert \alpha \neq \epsilon \}$.

  Para criar da forma correta o conjunto de produções $P_1$, devemos
  repetir a seguinte regra: fazer $P_1 = P_1 \in \{ A \rightarrow
  \alpha_1\alpha_2 \}$ para todo $A \rightarrow \alpha \in P_1$ e $X
  \in N_\epsilon$ tal que $\alpha = \alpha_1X\alpha_2$ e
  $\alpha_1\alpha_2 \neq \epsilon$. Isso significa que devemos olhar
  para cada uma das produções previamente inseridas (na etapa de
  inicialização) em $P_1$ e que contém um não-terminal que está em
  $N_\epsilon$, e inserir uma outra produção cuja forma sentencial do
  corpo é idêntica a produção sendo considerada mas sem o não-terminal
  escolhido de $N_\epsilon$. Esta regra iterativa realiza, dentro
  desse algoritmo, a geração de todas as combinações possíveis do
  corpo da produção que contém pelo menos um não-terminal que deriva
  para vazio. Como o processo é iterativo, a cada vez que uma nova
  produção é adicionada em $P_1$, devemos analisá-la posteriormente
  através desta mesma regra.

\item {\bf Incluir a produção vazia se necessário}

  Este passo somente deve ser realizado se a palavra vazia fizer parte
  da linguagem gerada da gramática original. Neste caso, devemos
  adicionar a regra $S \flecha \epsilon$ ao conjunto $P_1$ do passo
  anterior, considerando que $S$ é o símbolo inicial. Teremos então
  finalmente a gramática $G_2 = (N, T, P_2, S)$ onde $P_2 = P_1 \cup
  \{ S \flecha \epsilon \}$ e $N$, $T$, e $S$ são idênticos àqueles da
  gramática G.
\end{enumerate}

Ilustramos o funcionamento deste algoritmo com a seguinte gramática:

\begin{center}
\begin{tabular}{lll}
S & \flech & B {\bf z} B \\
B & \flech & A A A \\
A & \flech & {\bf a} \\
A & \flech & $\epsilon$ \\
\end{tabular}
\end{center}

No primeiro passo, inicializamos $N_\epsilon$ com o único não-terminal
que deriva \emph{diretamente} para vazio pela produção $A \flecha
\epsilon$. Portanto, $N_\epsilon = \{ A \}$. Depois, analisamos as
outras produções onde o corpo pode derivar para vazio de forma
\emph{indireta}. Observando a produção $B \flecha AAA$, vemos que como
$A$ pode derivar para vazio, toda a forma sentencial $AAA$ também pode
se tornar vazia. Caso isto ocorra, vemos que $B$ pode se tornar
vazio. Por causa disso, adicionamos $B$ também em $N_\epsilon$,
obtendo $N_\epsilon = \{ A, B \}$. Olhando para a única produção com
$S$, vemos que $S$ nunca pode derivar para vazio, visto que na forma
sentencial do corpo da sua produção temos o terminal {\bf z} que nunca
se torna vazio. A produção $A \flecha {\bf a}$ não deriva para vazio
de forma indireta pela mesma razão: {\bf a} é um símbolo
terminal. Terminamos portanto este passo com $N_\epsilon = \{ A, B \}$
finalizando a análise de todas as produções.

No segundo passo deste exemplo, inicializamos $P_1$ com todas as
produções de $P$ menos aquelas que derivam de forma \emph{direta} para
vazio. Portanto, $P_1$ tem $S \flecha B {\bf z} B$, $B \flecha AAA$ e
$A \flecha {\bf a}$ (todas menos a produção $A \flecha
\epsilon$). Para terminarmos o processo, analisamos separadamente cada
uma dessas três produções que fazem parte agora de $P_1$ tendo em
vista os não-terminais que foram adicionados em $N_\epsilon$ no passo
anterior. Analisando a produção $S \flecha B {\bf z} B$, sabemos que
$B$ deriva para vazio (pois se encontra em $N_\epsilon$). Sendo assim,
devemos incluir a combinação $S \flecha {\bf z} B$ em $P_1$ pois
$\alpha_1 = \epsilon$ e $\alpha_2 = {\bf z} B$. Vejam que a forma
sentencial ${\bf z}B$ é a concatenção de $\alpha_1 \alpha_2$, como
descrito na regra acima. Repetimos o processo considerando a mesma
produção, mas com $\alpha_1 = B {\bf z}$ e $\alpha_2 = \epsilon$,
rendendo a adição da nova produção $S \flecha B {\bf z}$. Como o
processo é iterativo, estas duas novas produções adicionais a $P_1$
devem ser analisadas pelo mesmo algoritmo, pois contém em suas
produções um não-terminal que está em $N_\epsilon$. Considerando
portanto a produção $S \flecha B {\bf z}$, vemos que $\alpha_1 =
\epsilon$ e $\alpha_2 = {\bf z}$. Como devemos adicionar a
concatenação $\alpha_1 \alpha_2$ como corpo de uma produção cuja
cabeça é $S$, adicionamos a nova produção $S \flecha {\bf
  z}$. Realizando o mesmo para a produção $S \flecha {\bf z} B$,
terminamos com produções adicionais em $P_1$ a partir da produção $S
\flecha B {\bf z} B$: $S \flecha {\bf z} B$, $S \flecha B {\bf z}$ e
$S \flecha {\bf z}$. Estas quatro produções representam todas as
combinações possíveis de $B{\bf z}B$, como descrito na parte
introdutória desta seção. Para terminar, repetimos todo esse processo
considerando a produção $B \flecha AAA$, pois $A$ também está em
$N_\epsilon$. Uma vez isto terminado, teremos um $P_1$ com as
seguintes produções:

\begin{center}
\begin{tabular}{lll}
S & \flech & B {\bf z} B \\
S & \flech & B {\bf z} \\
S & \flech & {\bf z} B \\
S & \flech & {\bf z} \\
B & \flech & A A A \\
B & \flech & A A \\
B & \flech & A  \\
A & \flech & {\bf a} \\
\end{tabular}
\end{center}

Como a palavra vazia não faz parte da linguagem gerada pela gramática
original do exemplo acima, $P_1$ é final.

\section{Remoção da recursão à esquerda}

Ainda por ser escrito.

\section{Fatoração gramatical}

Ainda por ser escrito.

\end{document}
