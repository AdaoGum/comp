\documentclass{compiladores}
\begin{document}

\begin{center}
{\LARGE Lista de Exercícios \#12}\\
{\bf Geração de TAC, Checagem de Tipos, Expressões Booleanas, Controle de Fluxo}
\end{center}

\bigskip

\begin{listanumerada}
\item Determine os endereços relativos para os identificadores seguintes considerando o esquema de tradução abaixo:
  \begin{lista}
  \item \texttt{float x;}
  \item \texttt{record \{ float x; float y; \} p;}
  \item \texttt{record \{ int tag; float x; float y; \} q;}
  \end{lista}

  \medskip

  \begin{tabular}{lll}
    P  &  $\rightarrow$  &  M D                                                    \\
    M  &  $\rightarrow$  &  $\epsilon$ \texttt { \{ offset = 0; \} }                           \\
    D  &  $\rightarrow$  &  T id; \texttt { \{ top.put(id.lexeme, T.tipo, offset);   offset += T.width; \} } D$_1$         \\
    D  &  $\rightarrow$  &  $\epsilon$                                             \\
    T  &  $\rightarrow$  &  \textbf{record} `\{' X D `\}' Y                        \\
    X  &  $\rightarrow$  &  $\epsilon$ \texttt{ \{ Env.push(top); top = new Env();   Stack.push(offset); offset = 0; \} }       \\
    Y  &  $\rightarrow$  &  $\epsilon$ \texttt{ \{ T.tipo = record(top); T.width = offset; top = Env.pop(); offset = Stack.pop(); \}  }\\
  \end{tabular}

\item Utilize o esquema de tradução abaixo para construir diferentes tipos válidos:

 \medskip

  \begin{tabular}{lll}
    T  &  $\rightarrow$  &  B \texttt{ \{ t = B.type; w = B.width; \} } C                        \\
    B  &  $\rightarrow$  &  \textbf{int} \texttt{ \{ B.type = integer; B.width = 4; \} }         \\
    B  &  $\rightarrow$  &  \textbf{float}  \texttt{ \{ B.type = float; B.width = 8; \} }        \\
    C  &  $\rightarrow$  &  $\epsilon$ \texttt{ \{ C.type = t; C.width = w; \} }                 \\
    C  &  $\rightarrow$  &  \textbf{[ num ]} $C_1$ \texttt{ \{ array(num.value, $C_1$.type); C.width = num.value * $C_1$.width; \} }\\
  \end{tabular}

\item O que é avaliação em curto-circuito? Quais suas vantagens e desvantagens?

\item Qual a diferença entre avaliação numérica e por controle considerando expressões booleanas?

\item \label{x} Considerando o esquema de tradução para avaliação numérica de
  expressões booleanas abaixo, gere código intermediário as seguintes
  expressões lógicas:
  \begin{lista}
  \item a $<$ b and not c $>$ d
  \item a $<$ b or c $<$ d and e $<$ f
  \item not a $<$ b or not c $>$ d and x $<$ q
  \item not (a $<$ b or not c $>$ d) and x $<$ q

  \end{lista}
  \medskip
  \begin{tabular}{llll}
    E  &  $\rightarrow$  &  E$_1$ or E$_2$   &  \et{ \{ E.nome = temp(); }                                      \\
    &                 &                   &  \et{ \ \ \ gera(E.nome = $E_1$.nome or $E_2$.nome \} }             \\
    E  &  $\rightarrow$  &  E$_1$ and E$_2$  &  \et{ \{ E.nome = temp(); }                                         \\
    &                 &                   &  \et{ \ \ \ gera(E.nome = $E_1$.nome and $E_2$.nome \} }            \\
    E  &  $\rightarrow$  &  not E$_1$        &  \et{ \{ E.nome = temp(); }                                         \\
    &                 &                   &  \et{ \ \ \ gera(E.nome = not $E_1$.nome \} }                       \\
    E  &  $\rightarrow$  &  (E$_1$)          &  \et{ \{ E.nome = $E_1$.nome \} }                                   \\
    E  &  $\rightarrow$  &  E$_1$ op E$_2$   &  \et{ \{ E.nome = temp(); }                                         \\
    &                 &                   &  \et{ \ \ \ gera(if $E_1$.nome op.simb $E_2$.nome goto proxq+3); }  \\
    &                 &                   &  \et{ \ \ \ gera(E.nome = 0); }                                     \\
    &                 &                   &  \et{ \ \ \ gera(goto proxq+2); }                                   \\
    &                 &                   &  \et{ \ \ \ gera(E.nome = 1); \} }                                  \\
    E  &  $\rightarrow$  &  true             &  \et{ \{ E.nome = temp();  }                                        \\
    &                 &                   &  \et{ \ \ gera(E.nome = 1); \} }                                    \\
    E  &  $\rightarrow$  &  false            &  \et{ \{ E.nome = temp();  }                                        \\
    &                 &                   &  \et{ \ \ gera(E.nome = 0); \} }                                    \\
  \end{tabular}

\item Avalie as expressões do exercício~\ref{x} considerando
  o esquema para avaliação por fluxo de controle abaixo.

  \medskip

  \begin{tabular}{lll}
    B  &  $\rightarrow$  &  \et{ \{ $B_1$.t=B.t; $B_1$.f=rot(); \} } B$_1$ or \et{ \{ $B_2$.t=B.t; $B_2$.f=B.f; \} } B$_2$   \\
    &                 &  \et{ \{ B.code=$B_1$.code $\vert\vert$\ label($B_1$.f) $\vert\vert$\ $B_2$.code \} }                 \\
    B  &  $\rightarrow$  &  \et{ \{ $B_1$.t=rot(); $B_1$.f=B.f; \} } B$_1$ and \et{ \{ $B_2$.t=B.t; $B_2$.f=B.f; \} } B$_2$  \\
    &                 &  \et{ \{ B.code=$B_1$.code $\vert\vert$\ label($B_1$.t) $\vert\vert$\ $B_2$.code \} }                 \\
    B  &  $\rightarrow$  &  not \et{ \{ $B_1$.t=B.f; $B_1$.f=B.t; \} } B$_1$ \et{ \{ B.code=$B_1$.code; \} }                 \\
    B  &  $\rightarrow$  &  (B$_1$) \et{ \{ B.code=$B_1$.code; B.t=$B_1$.t; B.f=$B_1$.f; \} }                                \\
    B  &  $\rightarrow$  &  true \et{ \{ B.code=gera(goto B.t); \} }                                                         \\
    B  &  $\rightarrow$  &  false \et{ \{ B.code=gera(goto B.f); \} }                                                        \\
    B  &  $\rightarrow$  &  E$_1$ relop E$_2$ \et{ \{ B.code=$E_1$.code $\vert\vert$ $E_2$.code $\vert\vert$ }               \\
    &                 &  \et{ gera(if $E_1$.local relop.lexval $E_2$.local goto B.t) $\vert\vert$ }                       \\
    &                 &  \et{ gera(goto B.f); \} }                                                                        \\
  \end{tabular}
\end{listanumerada}

\bigskip

Considerando o esquema de tradução abaixo:

\medskip

\begin{tabular}{lll}
 S  &  $\rightarrow$  &  attr \et{ \{ S.code=gera(attr.lexval) $\vert\vert$ gera(goto S.next) \} }                        \\
 S  &  $\rightarrow$  &  if  \et{ \{ B.t=rot(); B.f=S.next; \} }                                            \\
    &                 &  (B) \et{ \{ $S_1$.next=S.next; \} }                                                \\
    &                 &  S$_1$ \et{ \{ S.code=B.code $\vert\vert$ gera(B.t:) $\vert\vert$ $S_1$.code  \} }  \\
 S  &  $\rightarrow$  &  if  \et{ \{ B.t=rot(); B.f=rot(); \} }                                                           \\
    &                 &  (B) \et{ \{ $S_1$.next=S.next; \} }                                                              \\
    &                 &  S$_1$ else \et{ \{ $S_2$.next=S.next; \} }                                                       \\
    &                 &  S$_2$ \et{ \{ S.code=B.code $\vert\vert$ gera(B.t:) $\vert\vert$ $S_1$.code $\vert\vert$ }       \\
    &                 &  \et{ \ \ \ \ \ gera(B.f:); $\vert\vert$ $S_2$.code \} }                                          \\
 S  &  $\rightarrow$  &  while \et{ \{ B.f=S.next; B.t=rot(); \} }                                                        \\
    &                 &  (B) \et{ \{ S.begin=rot(); $S_1$.next=S.begin; \} }                                              \\
    &                 &  S$_1$ \et{ \{ S.code=gera(S.begin:) $\vert\vert$ B.code $\vert\vert$ }                           \\
    &                 &  \et{\ \ \ \ \ gera(B.t:) $\vert\vert$ $S_1$.code $\vert\vert$ gera(goto S.begin)  \} }           \\
 B  &  $\rightarrow$  &  \et{ \{ $B_1$.t=B.t; $B_1$.f=rot(); \} } B$_1$ or \et{ \{ $B_2$.t=B.t; $B_2$.f=B.f; \} } B$_2$   \\
    &                 &  \et{ \{ B.code=$B_1$.code $\vert\vert$\ label($B_1$.f) $\vert\vert$\ $B_2$.code \} }                 \\
 B  &  $\rightarrow$  &  \et{ \{ $B_1$.t=rot(); $B_1$.f=B.f; \} } B$_1$ and \et{ \{ $B_2$.t=B.t; $B_2$.f=B.f; \} } B$_2$  \\
    &                 &  \et{ \{ B.code=$B_1$.code $\vert\vert$\ label($B_1$.t) $\vert\vert$\ $B_2$.code \} }                 \\
 B  &  $\rightarrow$  &  not \et{ \{ $B_1$.t=B.f; $B_1$.f=B.t; \} } B$_1$ \et{ \{ B.code=$B_1$.code; \} }                 \\
 B  &  $\rightarrow$  &  (B$_1$) \et{ \{ B.code=$B_1$.code; B.t=$B_1$.t; B.f=$B_1$.f; \} }                                \\
 B  &  $\rightarrow$  &  true \et{ \{ B.code=gera(goto B.t); \} }                                                         \\
 B  &  $\rightarrow$  &  false \et{ \{ B.code=gera(goto B.f); \} }                                                        \\
 B  &  $\rightarrow$  &  E$_1$ relop E$_2$ \et{ \{ B.code=$E_1$.code $\vert\vert$ $E_2$.code $\vert\vert$ }               \\
    &                 &  \et{ gera(if $E_1$.local relop.lexval $E_2$.local goto B.t) $\vert\vert$ }                       \\
    &                 &  \et{ gera(goto B.f); \} }                                                                        \\
\end{tabular}

\medskip

Gere o código TAC para os trechos de código seguintes:

\begin{listanumerada}
\item 
  \begin{lstlisting}
    if (not (a < b or not c > d) and x < q) {
       x = z;
    }
  \end{lstlisting}

\item 
  \begin{lstlisting}
    while (a < b && e != f) {
      if (c < d){
        x = y + z;
      }else{
        x = x - z;
      }
    }  
  \end{lstlisting}

\item
  \begin{lstlisting}
    if (x > a){
      x = a;
    }else{
      x = q;
    }
  \end{lstlisting}
\end{listanumerada}

\medskip

Altere o esquema de tradução para fluxo de controle acima e adicione
regras de tradução para construções do tipo:
\begin{listanumerada}
\item 
  \begin{lstlisting}
    if (not (a < b or not c > d) and x < q) {
      x = z;
    }else if (x > b){
      x = k;
    }else if (a > e){
      x = q;
    }
  \end{lstlisting}

\item
  \begin{lstlisting}
    if (x > a){
      x = a;
    }else if (x > b){
      x = k;
    }else{
      x = q;
    }
  \end{lstlisting}

\item 
  \begin{lstlisting}
    for (i = x; x < a; s = a){
      if (x > a){
        x = a;
      }
    }
  \end{lstlisting}

\item 
  \begin{lstlisting}
    switch (a){
      case d: x = c;
      case b: x = b;
      default: x = c;
    }
  \end{lstlisting}
\end{listanumerada}

Traduza para TAC o código acima utilizando o esquema de tradução modificado.

\end{document}
