\documentclass{compiladores}
\begin{document}

\begin{center}
{\LARGE Lista de Exercícios \#02}\\
{\bf Gramáticas Livres de Contexto e Transformações}
\end{center}

\begin{listanumerada}
\item
Quais os principais objetivos do tratamento de erros da análise sintática?

\item 
Explique as três principais estratégias de tratamento de erros. Diga
uma vantagem de cada uma delas.

\item
O que é uma Gramática Livre de Contexto? O que a faz descrever a maioria das linguagens de programação?

\item
\label{execz}
Considerando a gramática livre de contexto descrita pelo conjunto de
produções $P = \{ S \rightarrow S\ S\ +\ |\ S\ S\ *\ |\ a\ \}$ e a
cadeira de entrada $aa + a*$.
\begin{lista}
\item Dê uma derivação mais à esquerda para a cadeia.
\item Dê uma derivação mais à direita para a cadeia.
\item Dê uma árvore de derivação para a cadeia.
\item A gramática é ambígua? Justifique a sua resposta.
\item Descreva textualmente a linguagem gerada por essa gramática.
\end{lista}

\item
Repita o exercício~\ref{execz} para cada uma das seguintes gramáticas e cadeias.
\begin{lista}
\item $S \rightarrow 0\ S\ 1\ |\ 0\ 1$ com as cadeias $000111$ e $01011001$. 
\item $S \rightarrow +\ S\ S\ |\ *\ S\ S\ |\ a$ com as cadeiras $+*aaa$ e $+a*aa$.
\item $S \rightarrow S\ (\ S\ )\ S\ |\ {\epsilon}$ com as cadeias $(()())$ e $((()())()())$.
\item $S \rightarrow S + S\ |\ SS\ |\ (\ S\ )\ |\ S\ *\ |\ a$ com as cadeias $(a+a)*a$ e $(a*a)a$.
\item $S \rightarrow (\ L\ )\ |\ a$ e $L \rightarrow L, S\ |\ S$ com a cadeia $((a,a),a,(a))$.
\item $S \rightarrow aSbS | bSaS | {\epsilon}$ com a cadeia $aabbab$.
\end{lista}

\item
\label{execy}
Considerando a gramática definida pelo conjunto de produções $P = \{ E
\rightarrow E - E\ |\ E + E\ |\ E * E\ |\ -E\ |\ +E\ |\ ( E )\ |\ id\ \}$,
elabore as sequências de derivações mais à esquerda e mais à direita para
as seguintes entradas:
\begin{lista}
\item $-(id+id*(id+id)*id)$
\item $(id)+(id)-(id*id)+(id+id)$
\item $((id*id)+(id+id))-(id)$
\item $(id+id-(id*id+id))+((id*id)-(id*id*id))$
\end{lista}

\item
Mostre que a gramática do exercício~\ref{execy} é ambígua utilizando
qualquer entrada válida para a gramática.

\item
Sugira uma gramática alternativa aquela do exercício~\ref{execy} mas
que não seja ambígua e que aceita exatamente a mesma linguagem gerada.

\item
\label{execx}
A gramática $G = (N, NT, P, S)$, o conjunto de não-terminais $NT = \{
S, L, E \}$, o conjunto de terminais $T = \{ begin, end, id, ;, :=,
\vee, \neg \}$ e as produções $P = \{ S \rightarrow
begin\ L\ end\ |\ id := E, L \rightarrow L\ ;\ S\ |\ S, E \rightarrow
E \vee E\ |\ \neg E\ |\ id\ \}$ é ambígua? Suporte a sua resposta com
argumentos.

\item
Define uma $G'$ não ambígua e equivalente à gramática $G$ do
  exercício \ref{execx}.

\item
Elimine a recursividade à esquerda das produções de $G$ do
  exercício \ref{execx}.

\item
Quais as duas principais estratégias de análise sintática? Diferencie-as conceitualmente e com um exemplo.

\item
A gramática a seguir define expressões regulares sob os símbolos $a$ e
$b$, usando o operador $+$ no lugar do operador $|$ para a união, a
fim de evitar o conflito com a barra vertical usada como um
meta-símbolo nas gramáticas:
\begin{tabular}{rclcrcl}
$rexpr$ & $\rightarrow$ & $rexpr + rterm\ |\ rterm$ & \hfill & $rterm$ & $\rightarrow$ & $rterm rfactor\ |\ rfactor$ \\
$rfactor$ & $\rightarrow$ & $rfactor *\ |\ rprimary$ & \hfill & $rprimary$ & $\rightarrow$ & $a\ |\ b$
\end{tabular}
\begin{lista}
\item Fatore esta gramática à esquerda.
\item A fatoração à esquerda torna a gramática adequada para a análise sintática descendente?
\item Além da fatoração à esquerda, elimine a recursão à esquerda da gramática original.
\item A gramática resultante é adequada para a análise sintática descendente?
\end{lista}

\item Considerando a gramática $S \rightarrow AB$, $A \rightarrow
  c|{\epsilon}$ e $B \rightarrow cbB|ca$, aplique as estratégias
  ascendente e descendente de análise para cada entrada: $ccbca$,
  $ca$, $cbca$, $cca$, $cbcbcbcbca$ e $ccbcbca$.
\end{listanumerada}
\end{document}

